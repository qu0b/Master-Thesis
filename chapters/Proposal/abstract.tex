% !TEX root = ../../proposal.tex
\begin{center}
  \textsc{Abstract}
\end{center}
%
\noindent
%

%%%%%%%%%%%%%%%%%%%%%%%%%%%%%%%%%%%%%%%%%%%%%%%%%%%%%%%%%%%%%%%%%%%%%%%%
% Abstract
%%%%%%%%%%%%%%%%%%%%%%%%%%%%%%%%%%%%%%%%%%%%%%%%%%%%%%%%%%%%%%%%%%%%%%%%

The increase in supply of on-demand computing resources, and the ever-growing number of internet capable devices calls for a new approach in provisioning cloud computing resources. A lot of research has gone into Service Level Agreements (SLA), from how they are defined, negotiated, and monitored to how they evolve. However, seeing as trust is currently a necessity for SLA negotiations, researchers have not yet looked at ways to remove the need for trust. This paper demonstrates that an agreement between a consumer and a provider can be reached, without either having to trust the other. We spotlight a method, which lets devices autonomously negotiate for cloud resources, on a trustless decentralized marketplace. Our focus lies on storing SLA on a Blockchain. That way in hindsight, SLA's can no longer be falsified. This work opens the doors to problematically integrating smart contracts into the negotiation process.

%Decentralized utility computing will become a reality. Currently, autonomous negotiation requires trust between provider and consumer. In this paper we look at how to make deals without the need for trust. Anyone will be able to sell their computing capacity on a utility computing marketplace.


%In this paper we show that it is possible and practical use a blockchain as a source of truth when negotiating web service agreements. Specifically we look look at bidirectional negotiation between a provider and consumer.