%%%%%%%%%%%%%%%%%%%%%%%%%%%%%%%%%%%%%%%%%%%%%%%%%%%%%%%%%%%%%%%%%%%%%%%%
\chapter{Contracts}
%%%%%%%%%%%%%%%%%%%%%%%%%%%%%%%%%%%%%%%%%%%%%%%%%%%%%%%%%%%%%%%%%%%%%%%%

In this chapter we will look at what exactly a contract is, how it is used and if a contract can be "smart". Further, we will analyze if a legal contract can be compared to a smart contract. How the contracts differ and

\section{Legal Contract}
A legal contract is defined by law, the definition can vary from country to country. It can be summed up as a binding agreement between two or more parties. Not abiding by the contract has legal consequences.

The typical form of a legal contract is a document with terms and signatures of the participants. These terms are not enforced by the contract itself, just defined in it. The terms of a contract are usually not defined in the contract itself but rather the terms reference the law. 

The contract itself does not enforce any of the terms. Rather, if a participant is accused of breaching the contract a judge will decide over the consequences. A smart contract on the other hand the judge and the terms.

\section{Smart Contract}
In this section we will look more at what exactly a smart contract is. Both from a technical and legal standpoint. Lets start with the name, it conveys to us that it is a form of contract more intelligent than a classical contract. From the definition above So is it a binding agreement.

